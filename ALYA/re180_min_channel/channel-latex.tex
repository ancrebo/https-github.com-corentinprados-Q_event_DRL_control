\documentclass[10pt]{article}
\usepackage[dvips]{epsfig}
\begin{document}
\title{Sum up of Alya run for problem: chan180}
\maketitle
\section{Problem definition}
\subsection{Physical problem}
In this run, Alya solves the modules NASTIN,KERMOD,
according to the following coupling strategy:
\begin{itemize}
  \item[] {\bf do}      1000000 time iterations 
  \begin{itemize}
\item[] block 1: {\bf do}   1 coupling iterations
\begin{itemize}
  \item[-] Solve module NASTIN
\end{itemize}
\item[] {\bf enddo} coupling iteration
  \end{itemize}
\item[] {\bf enddo} time iteration
\end{itemize}
The time interval is $[
0.000000E+00
,
0.100000E+01
]$.
Let $\delta t_{{\rm cri},i}$ be the critical time steps of module $i$,
and let $\alpha_{{\rm saf},i}$ be the safety factor of module $i$.
The time step $\delta t$ is taken as the minimum of the module critical time steps
multiplied by the corresponding safety factor such that
\begin{eqnarray*}
\delta t = \min_{i} \alpha_{{\rm saf},i} \delta t_{{\rm cri},i}
\quad \mbox{for all modules $i$ solved}
\end{eqnarray*}
\subsection{Geometrical data}
The computational domain belong to
$R^3$.
Let npoin be the number of nodes, nelem the number
of voulme elements and nboun the number of boundary elements.
We have:
\begin{eqnarray*}
 \mbox{npoin} &=&        0 \\
 \mbox{nelem} &=&        0 \\
 \mbox{nboun} &=&        0
\end{eqnarray*}
The domain data are
\begin{itemize}
   \item Total volume= 0.427200E+01
   \item Averaged element volume= 0.965712E-05
   \item Minimum element volume= 0.165228E-05
         for element 204097
   \item Maximum element volume= 0.192611E-04
         for element 39840
\end{itemize}
The elements that compose the volume and boundary meshes, as well
as their corresponding integration rules and numbers of Gauss points
is given in the following table.
\begin{center}
\begin{math}
\begin{array}{lllllll}
\hline
\textrm{Element} & \textrm{Total \#}  & \textrm{Dimension} & \textrm{Nodes \#} & 
\textrm{Integr.} & \textrm{Gauss}     & \textrm{Laplacian} \\
\textrm{}        & \textrm{}          & \textrm{}          & \textrm{} & 
\textrm{rule}    & \textrm{points \#} & \textrm{} \\
\hline
\textrm{QUA04} &      4608 & 2 &  4 & \textrm{open} &  4 & \textrm{no}\\
\textrm{HEX08} &    442368 & 3 &  8 & \textrm{open} &  8 & \textrm{no}\\
\hline
\end{array}
\end{math}
\mbox{Table: Volume and boundary elements}
\end{center}
\section{Module data}
\subsection{NASTIN: incompressible Navier-Stokes equations}
\subsubsection{Physical problem}
The equations solved are the
transient Navier-Stokes equations.
They read:
\begin{eqnarray*}
\rho \frac{\partial \mbox{\boldmath $u$}}{\partial t}
+ \rho [(\mbox{\boldmath $u$} \cdot \nabla)\mbox{\boldmath $u$}\mbox{\boldmath $u$} (\nabla \cdot \mbox{\boldmath $u$})]
-\nabla \cdot [2\mu\mbox{\boldmath $\varepsilon$}(\mbox{\boldmath $u$)}] 
+ \nabla p 
=\mbox{\boldmath $0$}
\\ \qquad
\nabla \cdot \mbox{\boldmath $u$} = 0
\end{eqnarray*}
The physical properties are
\begin{eqnarray*}
    \rho &=& 0.100000E+01
 \\ \mu  &=& 0.100000E+01
\end{eqnarray*}
\subsubsection{Numerical treatment}
The different ingredients of the numerical treatment are:
\begin{itemize}
  \item Spatial treatment:
  \begin{itemize}
    \item Stabilization: Algebaric SubGrid Scale (ASGS)
 without convection tracking
 and without time tracking
 and where the stabilization parameters are
          \begin{eqnarray*}
\tau_1 &=& \frac{1}{{\displaystyle c_1 \frac{\mu}{h^2}}+ {\displaystyle c_2 \rho \frac{|\mbox{\boldmath $u$}|}{h_c}}} \\\tau_2 &=& \frac{c_4}{c_1 \mu +c_2 \rho h_c |\mbox{\boldmath $u$}|}
          \end{eqnarray*}
          with $h$ and $h_c$ being the minimum element lengths
          with the following values for the constants
          \begin{eqnarray*}
              c_1 &=& 0.100000E+01
           \\ c_2 &=& 0.100000E+01
           \\ c_3 &=& 0.100000E+01
           \\ c_4 &=& 0.100000E+01
          \end{eqnarray*}
  \end{itemize}
  \item Temporal treatment:
  \begin{itemize}
    \item Time integration strategy:     \item Order of integration: 2
\end{itemize}
\end{itemize}
\section{Results}
The global convergence is shown in Figure \ref{fig:global-convergence}.
\begin{figure}
\centerline{\psfig{figure=latex-cvg.ps,width=0.95\textwidth}}
\caption{Global convergence.}
\label{fig:global-convergence}
\end{figure}
NASTIN convergence is shown in Figure \ref{fig:NASTIN-convergence}.
NASTIN solvers numbers of iterations is shown in Figure \ref{fig:NASTIN-solver}.
\begin{figure}
\centerline{\psfig{figure=latex-nastin-cvg.ps,width=0.95\textwidth}}
\caption{NASTIN convergence.}
\label{fig:NASTIN-convergence}
\end{figure}
\begin{figure}
\centerline{\psfig{figure=latex-nastin-sol.ps,width=0.95\textwidth}}
\caption{NASTIN solvers numbers of iterations.}
\label{fig:NASTIN-solver}
\end{figure}
\end{document}
